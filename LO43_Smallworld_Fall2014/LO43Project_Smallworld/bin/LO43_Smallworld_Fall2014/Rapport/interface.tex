Nous avons décidé de partitionner notre interface en 4 panneaux :
		\begin{description}
			\item[-]Un pour le plateaux de jeu
			\item[-]Un pour les actions du joueur
			\item[-]Un pour la pioche
			\item[-]Un pour rappeler l'ordre des actions
		\end{description}	
		
		Le plateau sera une image de fond avec des éléments cliquables dessus où apparaitront les peuples. Les actions joueurs et la pioche ne seront que des éléments clicables, alors	que le panneau d'information ne sera qu'une succession d'élements textuels à caractères purement informatif.  
	\section{Solutions techniques}
		Nous avions commencé une implementation de l'interface en utlisant la bibliothèque graphique Swing. Mais cette bibliothèque ne nous convient pas assez, elle est lourde à mettre en place pour avoir des éléments dynamique, et elle n'est maintenue que pour des correctifs de bug, et donc ne propose plus d'avancées technologiques. 
		
		Pour ces raisons, nous sommes en train d'étudier une migration vers JavaFX. De plus la communauté autour de cette bibliothèque semble très active sur le net, et donc offre un soutien et des tutoriels. Elle offre aussi la possibilité de travailler avec des formats d'image en plus que ceux disponible sur swing (es : svg) permettant de travailler avec des logiciels orientés vers l'image tel que Photoshop\up{\copyright}, Illustrator\up{\copyright}, ou encore Gimp\up{\copyright}. 
		
		Elle a aussi l'avantage d'être très portable (Linux, Mac, Windows, ou encore Android). Cela permet d'éventuellement penser à un portage multiplateforme.